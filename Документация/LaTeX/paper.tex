\documentclass[a4paper,12pt]{article}

%%% Работа с русским языком
\usepackage{cmap}					% поиск в PDF
\usepackage{mathtext} 				% русские буквы в формулах
\usepackage[T2A]{fontenc}			% кодировка
\usepackage[utf8]{inputenc}			% кодировка исходного текста
\usepackage[english, main=russian]{babel}	% локализация и переносы
%\usepackage[american,russian]{babel}

\usepackage{indentfirst}
\usepackage{lastpage} % Узнать, сколько всего страниц в документе.
\usepackage{multicol} % Несколько колонок
\usepackage{soul} % Модификаторы начертания

\frenchspacing

\usepackage[babel]{csquotes} % Еще инструменты для ссылок
%\usepackage[backend=biber,sorting=ynt,maxcitenames=2,style=verbose]{biblatex}
%\usepackage[citestyle=apa,sortcites=true,bibencoding=utf8,sorting=nyt,backend=biber]{biblatex}%
%\usepackage[citestyle=apa,backend=biber]{biblatex}
\usepackage[style=apa,citestyle=apa,backend=biber]{biblatex}
%style=verbose,
\DeclareLanguageMapping{american}{american-apa}
\DeclareLanguageMapping{russian}{american-apa}
\addbibresource{sources.bib}

\defbibfilter{papers}{
  type=article or
  type=book
}

%\renewcommand{\refname}{Список литературы}  % По умолчанию "Список литературы" (article)
%\renewcommand{\bibname}{Литература}  % По умолчанию "Литература" (book и report)



% --------------------------------------------------------------------------
% Пакет для заполнение точками оглавления
\usepackage{tocloft} 
%\renewcommand{\cftdot}{.} %empty {} for no dots. you can have any symbol inside. For example put {\ensuremath{\ast}} and see what happens.
%\renewcommand{\cftpartleader}{\cftdotfill{\cftdotsep}} % for parts
%\renewcommand{\cftchapleader}{\cftdotfill{\cftdotsep}} % for chapters
\renewcommand{\cftsecleader}{\cftdotfill{\cftdotsep}} % for sections, if you really want! (It is default in report and book class (So you may not need it).
\renewcommand{\cftsubsecleader}{\cftdotfill{\cftdotsep}}
% --------------------------------------------------------------------------
%\renewcommand{\thesection}{}
\renewcommand{\thesubsection}{}
\setcounter{tocdepth}{3} %level -1: part, 0: chapter, 1: section, 2: subsection etc.







\usepackage{hyperref}
\usepackage[usenames,dvipsnames,svgnames,table,rgb]{xcolor}
\hypersetup{				% Гиперссылки
	unicode=true,           % русские буквы в раздела PDF
	pdftitle={Задание 4},   % Заголовок
	pdfauthor={Олег Ящук},      % Автор
	pdfsubject={Тема},      % Тема
	pdfcreator={Создатель}, % Создатель
	pdfproducer={Производитель}, % Производитель
	pdfkeywords={keyword1} {key2} {key3}, % Ключевые слова
	colorlinks=true,       	% false: ссылки в рамках; true: цветные ссылки
	linkcolor=black,        % внутренние ссылки
	citecolor=black,        % на библиографию
	filecolor=magenta,      % на файлы
	urlcolor=blue           % на URL
}

%%% Программирование
\usepackage{etoolbox} % логические операторы

%%% Страница
\usepackage{extsizes} % Возможность сделать 14-й шрифт
\usepackage{geometry} % Простой способ задавать поля
	\geometry{top=30mm}
	\geometry{bottom=40mm}
	\geometry{left=30mm}
	\geometry{right=20mm}

%\usepackage{fancyhdr} % Колонтитулы
% 	\pagestyle{fancy}
%	\renewcommand{\headrulewidth}{0pt}  % Толщина линейки, отчеркивающей верхний колонтитул
% 	\lfoot{Нижний левый}
% 	\rfoot{Нижний правый}
% 	\rhead{Верхний правый}
% 	\chead{Верхний в центре}
% 	\lhead{Верхний левый}
%	\cfoot{Нижний в центре} % По умолчанию здесь номер страницы

\usepackage{setspace} % Интерлиньяж
\onehalfspacing % Интерлиньяж 1.5
%\doublespacing % Интерлиньяж 2
%\singlespacing % Интерлиньяж 1

%%% Работа с картинками
\usepackage{graphicx}  % Для вставки рисунков
%\graphicspath{{images/}{images2/}}  % папки с картинками
\setlength\fboxsep{3pt} % Отступ рамки \fbox{} от рисунка
\setlength\fboxrule{1pt} % Толщина линий рамки \fbox{}
\usepackage{wrapfig} % Обтекание рисунков текстом

%%% Работа с таблицами
\usepackage{array,tabularx,tabulary,booktabs} % Дополнительная работа с таблицами
\usepackage{longtable}  % Длинные таблицы
\usepackage{multirow} % Слияние строк в таблице

\usepackage{tikz} % Работа с графикой
\usepackage{pgfplots}
\usepackage{pgfplotstable}

%%% Дополнительная работа с математикой
\usepackage{amsmath,amsfonts,amssymb,amsthm,mathtools} %% AMS
\usepackage{icomma} % "Умная" запятая: $0,2$ --- число, $0, 2$ --- перечисление

%% Номера формул
%\mathtoolsset{showonlyrefs=true} % Показывать номера только у тех формул, на которые есть \eqref{} в тексте.
%\usepackage{leqno} % Нумерация формул слева

%% Свои команды
\DeclareMathOperator{\sgn}{\mathop{sgn}}

%% Перенос знаков в формулах (по Львовскому)
\newcommand*{\hm}[1]{#1\nobreak\discretionary{}
	{\hbox{$\mathsurround=0pt #1$}}{}}

%%% Теоремы
\theoremstyle{plain} % Это стиль по умолчанию, его можно не переопределять.
\newtheorem{theorem}{Теорема}[section]
\newtheorem{proposition}[theorem]{Утверждение}

\theoremstyle{definition} % "Определение"
\newtheorem{corollary}{Следствие}[theorem]
\newtheorem{problem}{Задача}[section]

\theoremstyle{remark} % "Примечание"
\newtheorem*{nonum}{Решение}


\author{O.O.Yashchuk} %, I.S.Bondarenko
\title{Multiple hypothesis testing in the task of AB testing}
\date{\today}

%\usepackage{lipsum} 

%\newcommand*{\rom}[1]{\expandafter\@slowromancap\romannumeral #1@}
\newcommand{\rom}[1]{\uppercase\expandafter{\romannumeral #1\relax}}
\begin{document} % конец преамбулы, начало документа

\maketitle

\begin{abstract}
%	\lipsum[1]
Sequential testing with likelihood ratios which requires fewer observations than classical hypothesis testing is proposed. It allows to speed up A/B testing. Decisions about effectiveness of variations of web pages are made as soon as possible. Software implementation of A/B testing has been developed in R.
\end{abstract}

\tableofcontents

%\section{План исследовательской работы}
%\subsection{Рассмо}
%\subsection{AA тестировани}
%\subsection{AA тестировани}
%\subsection{AA тестировани}

\section{Введение}
\begin{itemize}
	\item Описать простую задачу АВ тестирования, \textbf{и её приминение на практике}.
	\item Способ решения простой задачи АВ тестирования.
	\item Цель исследования представленного в работе.
\end{itemize}

\section{Постановка задачи}
\subsection{Структура раздела}
\begin{itemize}
	\item Рассмотреть проблематику множественной проверки гиппотез. 
			\begin{itemize}
				\item Таблица ($V, R, m, m_0 ...$)
				\item Описание $FWER=P(V>1)$ и необходимости контроля $FWER\leq\alpha$
				\item $FDR$ загвоздка в нём???
				\item Небольшой пример?
			\end{itemize} 
	\item Для каких случаев множественной проверки гипотез растет вероятность ошибки \rom{1} рода? (Когда гипотезы проверяются на одних и тех же данных?)
	\item Показать, что при множественной проверке гиппотиз растет вероятность груповой ошибки \rom{1} рода $FWER$.
			\begin{itemize}
				\item Пример при помощи теореммы Пуассона \href{https://www.cs.cornell.edu/~asampson/blog/statsmistakes.html?utm_campaign=Data%2BElixir&utm_medium=email&utm_source=Data_Elixir_108}{идея отсюда}
				\item На графике показать рост вероятности ошибки \href{http://conversionxl.com/how-many-ab-test-variations/?utm_source=sumome&utm_medium=linkedin&utm_campaign=share}{вот как здесь}
				\item \textbf{Проверить на АА тестирование верность этого утверждения!}
			\end{itemize} 		
\end{itemize}

\subsection{title}
Вероятность допустить хотя-бы одну ошибку \rom{1} рода. Воспользуемся предельной теоремой Пуассона.
\begin{multline}
	P(V>k)= 
	1-\sum_{i=1}^{k}C_{n}^{k}\alpha^i(1-\alpha)^{n-i} = 
	1-\exp{-\lambda}\sum_{i=0}^{k}\frac{\lambda^i}{i!} 
	\text{, where } \lambda=n\alpha
\end{multline}

\textbf{Example.} Вероятность допустить хотя бы одну ошибку \rom{1} рода при 10 проверках гипотез. При заданом ограничении вероятности ошибки \rom{1} рода равной 5\%.

\begin{equation*}
P(V>0)=1-\exp^{-\lambda}\sum_{i=0}^{k}\frac{\lambda^i}{i!} =
1-\exp{-\frac{1}{2}} = 0.39 \equiv 39\%
\end{equation*}

\begin{equation*}
	\begin{aligned}	
		P(V>1)=&1-\exp{-\lambda}\sum_{i=0}^{k}\frac{\lambda^i}{i!} =
		1-\exp{-0.5}\sum_{i=0}^{1}\frac{0.5^i}{i!} =\\
		&1-\exp{-0.5}\left(\frac{0.5^0}{0!}+\frac{0.5^1}{1!}\right) = 
		1-\exp{-0.5}\left(1+0.5\right) = \\
		&1-1.5\exp{-0.5} = 0.09 \equiv 9\%
	\end{aligned}
\end{equation*}

\section{Класична перевірка статистичних гіпотез}
\subsection{Звичайне A/B-тестування}
\subsection{Множинне A/B-тестування}

\section{Послідовна перевірка статистичних гіпотез}
\subsection{Звичайне A/B-тестування}
\subsection{Множинне A/B-тестування}

\section{Порівняння класичної та послідовної перевірки статистичних гіпотез при множинному A/B-тестуванні}


%ref{https://help.optimizely.com/Build_Campaigns_and_Experiments/Experiment_Types%3A_AB%2C_Multivariate%2C_and_Multi-page#multivariate_test

\nocite{*}
\printbibliography[heading=subbibliography,filter=papers,title={Литература}]%type=book,
\end{document} % конец документа


