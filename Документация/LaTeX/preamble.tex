%%% Работа с русским языком
\usepackage{cmap}					% поиск в PDF
\usepackage{mathtext} 				% русские буквы в формулах
\usepackage[T2A]{fontenc}			% кодировка
\usepackage[utf8]{inputenc}			% кодировка исходного текста
\usepackage[english, main=russian]{babel}	% локализация и переносы
%\usepackage[american,russian]{babel}

\usepackage{indentfirst}
\usepackage{lastpage} % Узнать, сколько всего страниц в документе.
\usepackage{multicol} % Несколько колонок
\usepackage{soul} % Модификаторы начертания

\frenchspacing

\usepackage[babel]{csquotes} % Еще инструменты для ссылок
%\usepackage[backend=biber,sorting=ynt,maxcitenames=2,style=verbose]{biblatex}
%\usepackage[citestyle=apa,sortcites=true,bibencoding=utf8,sorting=nyt,backend=biber]{biblatex}%
%\usepackage[citestyle=apa,backend=biber]{biblatex}
\usepackage[style=apa,citestyle=apa,backend=biber]{biblatex}
%style=verbose,
\DeclareLanguageMapping{american}{american-apa}
\DeclareLanguageMapping{russian}{american-apa}
\addbibresource{sources.bib}

\defbibfilter{papers}{
  type=article or
  type=book
}

%\renewcommand{\refname}{Список литературы}  % По умолчанию "Список литературы" (article)
%\renewcommand{\bibname}{Литература}  % По умолчанию "Литература" (book и report)



% --------------------------------------------------------------------------
% Пакет для заполнение точками оглавления
\usepackage{tocloft} 
%\renewcommand{\cftdot}{.} %empty {} for no dots. you can have any symbol inside. For example put {\ensuremath{\ast}} and see what happens.
%\renewcommand{\cftpartleader}{\cftdotfill{\cftdotsep}} % for parts
%\renewcommand{\cftchapleader}{\cftdotfill{\cftdotsep}} % for chapters
\renewcommand{\cftsecleader}{\cftdotfill{\cftdotsep}} % for sections, if you really want! (It is default in report and book class (So you may not need it).
\renewcommand{\cftsubsecleader}{\cftdotfill{\cftdotsep}}
% --------------------------------------------------------------------------
%\renewcommand{\thesection}{}
\renewcommand{\thesubsection}{}
\setcounter{tocdepth}{3} %level -1: part, 0: chapter, 1: section, 2: subsection etc.







\usepackage{hyperref}
\usepackage[usenames,dvipsnames,svgnames,table,rgb]{xcolor}
\hypersetup{				% Гиперссылки
	unicode=true,           % русские буквы в раздела PDF
	pdftitle={Задание 4},   % Заголовок
	pdfauthor={Олег Ящук},      % Автор
	pdfsubject={Тема},      % Тема
	pdfcreator={Создатель}, % Создатель
	pdfproducer={Производитель}, % Производитель
	pdfkeywords={keyword1} {key2} {key3}, % Ключевые слова
	colorlinks=true,       	% false: ссылки в рамках; true: цветные ссылки
	linkcolor=black,        % внутренние ссылки
	citecolor=black,        % на библиографию
	filecolor=magenta,      % на файлы
	urlcolor=blue           % на URL
}

%%% Программирование
\usepackage{etoolbox} % логические операторы
